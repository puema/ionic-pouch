\chapter{PouchDB}
\label{PouchDB}

\section{Adapter}
\label{Adapter}

Laut eigener Dokumentation ist PouchDB keine eigenständige Datenbank. Stattdessen versteht sich PouchDB als Datenbankabstraktionsschicht und nutzt im Hintergrund andere Datenbanken. Der Zugriff auf diese Datenbanken wird durch sogenannte Adapter umgesetzt. Ziel dieser Abstraktion ist es eine einheitliche Programmierschnittstelle (API) zur Verfügung zu stellen, die in verschiedenen JavaScript-Umgebungen und Browsern gleich funktioniert.

PouchDB wird mit Adaptern für die Browser-Datenbanken IndexedDB und WebSQL sowie einem Adapter für LevelDB ausgeliefert. Weitere Adapter für SQLite, den Localstorage oder eine In-Memory-Datenbank stehen als Plugins zur Verfügung \cite{pouch:adapters}.


Bei dem in dieser Arbeit vorgestellten Technologiestack werden Apps im Browser von mobilen Geräten ausgeführt, deshalb soll der Einsatz von PouchDB im Browser genauer betrachtet werden.

Erzeugt man, wie im folgenden Codeausschnitt beschrieben, eine neue PouchDB-Datenbank, so  nutzt PouchDB im Browser vorzugsweise den IndexedDB-Adapter. Unterstützt der verwendete Browser das nicht, greift PouchDB automatisch auf den WebSQL-Adapter zurück \cite{pouch:adapters}.
\begin{codebox}
\begin{lstlisting}[style=typescript]
var pouch = new PouchDB('myDB');
\end{lstlisting}
\end{codebox}


Die Datenmengen, die in den Browserdatenbanken IndexedDB und WebSQL abgelegt werden können, sind in mobilen Browsern begrenzt \cite{html5:quota}. Um diese Begrenzungen zu umgehen empfiehlt die PouchDB-Dokumentation die Verwendung einer SQLite-Datenbank mit dem entsprechenden Adapter. Eine SQLite-Datenbank kann laut Dokumentation auch den Entwicklungsprozess erleichtern, da SQLite-Datenbanken als .sqllite-Dateien für Entwickler direkt zugänglich sind \cite{pouch:adapters}.

Im folgenden Codeausschnitt wird eine neue PouchDB-Datenbank mit dem Namen \texttt{db} erzeugt und manuell die Verwendung des SQLite-Adapter festgelegt \cite{pouch:adapters}.
\begin{codebox}
\begin{lstlisting}[style=typescript]
var db = new PouchDB('db', {adapter: 'cordova-sqlite'});
\end{lstlisting}
\end{codebox}


\section{API}
\label{API}
PouchDB kann als JavaScript-Implementierung von CouchDB angesehen werden. Das selbst erklärte Ziel von PouchDB ist es die CouchDB-API im Browser und in Node.js zu emulieren \cite{pouch:intro}. 

Die API von PouchDB ist laut eigener Dokumentation asynchron. Sie kann über Callbacks, Promises oder async-Funktionen angesprochen werden \cite{pouch:api}.


\section{Synchronisation}
\label{Synchronisation}

\begin{citeenv}
	\begin{quotation}
		"`The way I like to think about CouchDB is this: CouchDB is bad at everything, except syncing. And it turns out that's the most important feature you could ever ask for, for many types of software."' \cite{pouch:replication}.
	\end{quotation}
\end{citeenv}